\documentclass[compress,xcolor=table]{beamer}

% Packages
\usepackage[english]{babel}
\usepackage[utf8]{inputenc}
\usepackage[T1]{fontenc}
\usepackage{datetime}
%\usepackage{booktabs}

% Possible options of the package (add/remove below in \usetheme call):
%  - nosectionpages: no pages between sections
%  - flama: use flama font, requires xelatex/lualatex + the font to compile
%  - compressminiframes: put the heading list bullets indications pages on 1 line
\usetheme[compressminiframes]{sorbonne}

% Title page
\title{Introduction}
\foottitle{VLF Model} % optional, printed at the bottom of the slides, by default same as title, can be useful to rewrite when title has a newline for example
\subtitle{VLF Model} % optional subtitle
\date{\formatdate{22}{03}{2020}}
\author{Camilo Andrés Salazar González}
\institute{Physics Institute} % Optional

% Biblatex
\usepackage[backend=bibtex, style=authoryear, citestyle=authoryear]{biblatex}
\bibliography{library.bib}
\renewcommand*{\bibfont}{\footnotesize}


%%%%
%% BEGIN OF SLIDES
%%%%

\begin{document}

\begin{frame}[plain]
	\titlepage
	\setcounter{framenumber}{0}
\end{frame}

\begin{frame}[allowframebreaks]{Overview}
\tableofcontents
\end{frame}

%%%%%%%%%%%%%%%%%%%%%%%%%%%%%%%%%%%%%%%%%%%%%%%%%%%%%%%%%
\section{WIMP} 


\subsection{The WIMP paradigm}

\begin{frame}{}
Assuming there exist interactions between a cosmologically stable particle $\chi$-the (generic) DM with Standard Model (SM) particles, sizable enough that for a high enough temperature $T$ the DM is in thermal equilibrium with the primordial thermal bath, the cosmological evolution of the DM particle can be traced through the following Boltzmann equation\footcite{Arcadi2017}:
\[
\frac{dn_{\chi}}{dt}+3H(T)n_{\chi}=-\left\langle \sigma v \right\rangle \left(n^2_{\chi}-n^2_{\chi,eq}\right)
\]
describing the DM number density $n_{\chi}$ in turn defined as:

\[
n_{\chi}(T)=\int\frac{d^3p}{(2\pi)^2}f_{\chi}(p,T)
\]

\end{frame}


\begin{frame}{}


As the Universe expands, the scale factor increases and the temperature decreases. Assuming that $\chi$ continues to be in thermal equilibrium, eventually the temperature drops below the DM mass, and the annihilation rate for DM particles, which depends linearly on the number equilibrium density $n_{\chi,eq}$, enters the so-called “Boltzmann-tail”: 
\[
n_{\chi,eq}\propto\exp(-m_{\chi}/T)
\]; 
the annihilation rate then eventually fall below the Universe expansion rate, $H(T)$ (a power-law in temperature, $\propto T^2$ in the radiation-dominated universe), leading to the thermal freeze-out of this “cold” relic. Thereafter, the DM comoving number density $Y_{\chi}=\frac{n_{\chi}}{s}$ \footnote{text} is approximately constant.

\end{frame}

%%%%%%%%%%%%%%%%%%%%%%%%%%%%%%%%%%%%%%%%%%%%%%%%%%%%%%%%%
\section{Co-Annihilation}
\subsection{Co-Annihilation Mechanism} 

\begin{frame}{}

\end{frame}

%%%%%%%%%%%%%%%%%%%%%%%%%%%%%%%%%%%%%%%%%%%%%%%%%%%%%%%%%
\section{Dark Matter Models} 
\subsection{Single Scalar (SS)}
\subsubsection{Experiment Constrains}
\begin{frame}{}

\end{frame}


\subsection{}
\subsubsection{Experiment Constrains}

\subsection{VLF}
\subsubsection{Co-Annihilation}
\subsubsection{DM}
\subsubsection{Yukawa}
\subsubsection{Taus}
\subsubsection{Experiment Constrains}

\begin{frame}{Some papers for noisy images datasets}
	
	\begin{exampleblock}{\citetitle{Azadi2015} \cite{Azadi2015}}
		
		\begin{itemize}
			\item
			Define a regularized loss for training the CNN
			\item
			Can be seen as looking for the label of similar images for
			regularization
			\item
			Results slightly better than Sukhbaatar model
		\end{itemize}
		
	\end{exampleblock}

	\begin{alertblock}{The problem}
	
	\begin{itemize}
		\item
		UPMC Food-101 has been crawled from Google Images
		\item
		It contains a certain amount of noise
	\end{itemize}
	
	\end{alertblock}
	
\end{frame}

\section{References} \subsection{}

\begin{frame}[allowframebreaks]{References}
	
	\printbibliography[heading=none]
	
\end{frame}




\end{document}

